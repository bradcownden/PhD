\documentclass[../PhD.tex]{subfiles}

%\usepackage{jheppub}
%\usepackage{graphicx}
%\usepackage{hyperref}
%\usepackage{caption,subcaption} % Subfigures
%\usepackage[section]{placeins}
%\usepackage[utf8]{inputenc}
%\usepackage{cite}

%\newcommand{\comment}[1]{}
%\newcommand{\beq}[1]{\begin{equation}\label{#1}}
%\newcommand{\eeq}{\end{equation}}


%%%%%%%%%%%%%%
\begin{document}

\chapter{Nonlinear Evolution of Massive Scalar Fields}

Published to the arXiv, submitted to Phys Rev D.

\section{Contributions}

What my contributions were

\newpage

\begin{center}
{\bf{\Large Phase Diagram of Stability for Massive Scalars in Anti-de Sitter
Spacetime}} \\ 
\bigskip
{\bf \href{https://arxiv.org/abs/1711.00454}{arXiv:1711.00454}} \\
\bigskip
\bigskip
Brad Cownden$^1$, Nils Deppe$^2$, and Andrew R.~Frey$^{3,4}$\\
\bigskip

$^1${\it Department of Physics \& Astronomy,\\ University of Manitoba\\
Winnipeg, Manitoba R3T 2N2, Canada \\ {\rm cowndenb@myumanitoba.ca}} \\
\vspace{0.1in}

$^2${\it Cornell Center for Astrophysics and Planetary Science and
Department of Physics,\\ Cornell University\\
122 Sciences Drive, Ithaca, New York 14853, USA \\ {\rm nd357@cornell.edu}} \\
\vspace{0.1in}

$^3${\it Department of Physics \& Astronomy,\\ University of Manitoba\\
Winnipeg, Manitoba R3T 2N2, Canada} \\
$^4${\it Department of Physics and Winnipeg Institute for Theoretical
Physics,\\ University of Winnipeg\\
515 Portage Avenue, Winnipeg, Manitoba R3B 2E9, Canada \\ {\rm a.frey@uwinnipeg.ca}}
\end{center}

\bigskip

We present the phase diagram of stability of 5-dimensional
anti-de Sitter spacetime against horizon
formation in the gravitational collapse of a scalar field, treating the scalar
field mass and width of initial data as free parameters. We find that 
the stable phase becomes larger and shifts to smaller widths as the field
mass increases. In addition to
classifying initial data as stable or unstable, we identify two other phases
based on nonperturbative behavior.  The metastable phase forms a horizon over
longer time scales than suggested by the lowest order perturbation theory,
and the irregular phase can exhibit non-monotonic and chaotic behavior in
the horizon formation times. Our results include evidence for chaotic
behavior even in the collapse of a massless scalar field.

\section{Introduction}

Through the anti-de Sitter spacetime (AdS)/conformal field theory (CFT)
correspondence, string theory on AdS$_5\times X^5$ is
dual to a large $N$ conformal field theory in four spacetime dimensions
($\mathbb{R}\times S^3$ when considering global AdS$_5$).  The simplest
time-dependent system to study in this context is the gravitational dynamics
of a real scalar field with spherical symmetry, corresponding to the
time dependence of the expectation value of the zero mode of a single
trace operator in the gauge theory.  Starting with the pioneering work of
\cite{1104.3702,1106.2339,1108.4539,1110.5823}, numerical studies have
suggested that these dynamics may in fact be generically unstable
toward formation of (asymptotically) AdS$_{d+1}$ black holes
even for arbitrarily small amplitudes.  While perhaps surprising compared
to intuition from gravitational collapse in asymptotically flat spacetimes,
the dual picture of thermalization of small energies in a compact space
is more expected.  In terms of the scalar eigenmodes on a fixed AdS background,
the instability is a cascade of energy to higher frequency modes and shorter
length scales (weak turbulence), which eventually concentrates energy within
its Schwarzschild radius.  In a naive perturbation theory, this is evident
through secular growth terms.

However, some initial scalar field profiles lead to quasi-periodic evolution
(at least on the time scales accessible via numerical studies)
at small but finite amplitudes; even early work \cite{1104.3702,1109.1825}
noted that it is possible to remove the secular growth terms in the
evolution of a single perturbative eigenmode.  A more sophisticated
perturbation theory \cite{1403.6471,1407.6273,Basu:2014sia,1410.1880,1412.4761,1412.3249,Evnin:2015gma,1507.02684,1507.08261,1508.04943,1508.05474,1510.07836}
supports a broader class of quasi-periodic solutions that can contain
non-negligible contributions from many modes, and other stable solutions
orbit the basic quasi-periodic solutions \cite{1507.08261}.  Stable
solutions exhibit inverse cascades of energy from higher frequency to lower
frequency modes due to conservation laws following from the high symmetry
of AdS (integrability of the dual CFT).  Stable behavior also appears
in the full non-perturbative dynamics for initial profiles with widths near
the AdS length scale \cite{1304.4166,1307.2875,1308.1235}; however,
analyses of the perturbative and full dynamics in the literature have not
always been in agreement at fixed small amplitudes.  For example, some
perturbatively stable evolutions at finite amplitude actually form black holes
in numerical evaluation of the full dynamics
\cite{1403.6471,1410.2631,1506.07907}.  Understanding the breakdown of the
approximations used in the perturbative theory, as well as its region of
validity, is an active and important area of research
\cite{1506.03519,1606.02712,1607.08094,Dimitrakopoulos:2016euh,Liebling:2017gfn}.

Ultimately, the main goal of this line of inquiry is to determine whether
stability or instability to black hole formation (or both) is generic on
the space of initial data, so the extent of the ``islands of stability''
around single-mode or other quasi-periodic solutions and how it varies
with parameters of the physics on AdS are key questions of interest.  The
biggest changes occur in theories with a mass gap in the black hole spectrum,
such as AdS$_3$ and Einstein-Gauss-Bonnet gravity in AdS$_5$, which cannot
form horizons at small amplitudes.  While small-amplitude evolution in
AdS$_3$ appears to be quasi-periodic \cite{1306.0317,1412.6002}, there is
some evidence to point toward late-time formation of a naked singularity
in AdS$_5$ Einstein-Gauss-Bonnet gravity \cite{1608.05402,1410.1869}
(along with a power law energy spectrum similar to that at horizon
formation).  Charged scalar and gauge field matter \cite{1606.00830}
also introduces a qualitative change in that initial data may lead to
stable evolution or instability toward either Reissner-Nordstr\"om black holes
or black holes with scalar hair.

In this paper, we extend the study of massive scalar matter initiated in
\cite{1504.05203,1508.02709}.  Specifically, using numerical evolution of
the full gravitational dynamics, we draw the phase diagram of gravitational
collapse as a function of scalar field mass and initial scalar profile
width.  By considering the time to horizon formation as a function of
the initial profile's amplitude, we identify several different classes
of stable behavior and indicate them on the phase diagram.  Finally, we
analyze and characterize these different stable behaviors.  Throughout,
we work in AdS$_5$, due to its relevance to strongly coupled gauge theories
in four dimensions and because previous literature has indicated massless
scalars lead to greater instability than in AdS$_4$ (the main other case
considered), which makes the effects of the scalar field mass more visible.

We note briefly two caveats for the reader.  First, horizon formation
always takes an infinite amount of time on the AdS conformal boundary
due to the usual time dilation effects associated with horizons; this agrees
with the understanding of thermalization in the CFT as an asymptotic process.
Horizon formation times discussed in this paper correspond to an approximate
notion of horizon formation that we will describe below, but alternate
measures of thermalization may be of interest.  Second, the black holes
we discuss are smeared on the compact $X^5$ dimensions of the gravitational
side of the duality, as in most of the literature concerning stability of
AdS, and we are particularly interested in small initial amplitudes that
lead to black holes small compared to the AdS scale. As described in
\cite{hep-th/0202189,1502.01574,1509.07780}, small black holes in this
situation suffer a Gregory-Laflamme-like instability toward localization
on $X^5$ (which may in fact lead to formation of a naked singularity).  At
the same time, certain light stable solutions for charged scalars (boson
stars) are stable against localization on $X^5$ \cite{1509.00774}.  We
therefore provisionally assume that the onset of the Gregory-Laflamme-like
instability occurs only at horizon formation, not at any point of the
earlier horizon-free evolution.


The plan of this paper is as follows: in section \ref{s:review}, we review
the time scales associated with horizon formation with an emphasis on
the behavior of massive scalars and briefly discuss our methods.  Then,
in section \ref{s:phases}, we present the phase diagram of different
stability behaviors, and an attempt at heuristic analytic understanding 
appears in \ref{s:analysis}.
We close with a discussion of our results.



\section{Review}\label{s:review}
In this section, we review results on the stability of scalar field initial
data as well as our methods (following the discussion of \cite{1508.02709}).

\subsection{Massive scalars, stability, and time scales}

As in most of the literature, we work in Schwarzschild-like coordinates,
which have the line element (in asymptotic AdS$_{d+1}$)
\beq{metric}
  ds^2=\frac{1}{\cos^2(x)}\left(-Ae^{-2\delta}dt^2+
  A^{-1}dx^2+\sin^2(x)d\Omega^{d-1}\right)
\eeq
in units of the AdS scale.  In these coordinates, a horizon appears at
$A(x,t)=0$, but reaching zero takes an infinite amount of time (measured either
in proper time at the origin or in conformal boundary time); following
the standard approach, we define a horizon as having formed at the earliest
spacetime point (as measured by $t$) where $A$ drops below a specified
threshold defined in \S\ref{s:methods} below.  Of course, horizon formation
represents a coarse-grained description since the pure initial state of
the dual CFT cannot actually thermalize; a more precise indicator of
approximate thermalization may be the appearance of a power law energy spectrum
(exponentially cut off) in the perturbative scalar eigenmodes.  This
indicator is tightly associated with horizon formation (though see
\cite{1608.05402,1410.1869} for some counterexamples).

A key feature of any perturbative formulation of the gravitational
collapse is that deviations from $A=1,\delta=0$ appear at order $\epsilon^2$,
where $\epsilon$ is the amplitude of initial data.  As a result,
horizons can form only after a time $t\sim\epsilon^{-2}$; in the
multiscale perturbation theory of \cite{1403.6471,1407.6273,1410.1880,1412.4761,1412.3249,1507.02684,1507.08261,1508.04943,1508.05474,1510.07836},
there is in fact a scaling symmetry
$\epsilon\to\epsilon',t\to t(\epsilon/\epsilon')^2$ that enforces the
proportionality $t_H\propto \epsilon^{-2}$, where $t_H$ is the (approximate)
horizon formation time for unstable initial data at small amplitude.

\begin{figure}[!t]
\centering
\begin{subfigure}[t]{0.47\textwidth}
\includegraphics[width=\textwidth]{/Users/bradc/Documents/University_of_Manitoba/Thesis/PhD/Chapter1/figs/m0w15.pdf}
\caption{\textit{Stable} initial data for $\sigma=1.5$}
\label{f:m0w15}
\end{subfigure}\hfill
\begin{subfigure}[t]{0.47\textwidth}
\includegraphics[width=\textwidth]{/Users/bradc/Documents/University_of_Manitoba/Thesis/PhD/Chapter1/figs/m0w025fit.pdf}
\caption{\textit{Unstable} initial data for $\sigma=0.25$}
\label{f:m0w025}
\end{subfigure}
\begin{subfigure}[t]{0.47\textwidth}
\includegraphics[width=\textwidth]{/Users/bradc/Documents/University_of_Manitoba/Thesis/PhD/Chapter1/figs/m0w085fit.pdf}
\caption{\textit{Metastable} initial data for $\sigma=0.85$}
\label{f:m0w085}
\end{subfigure}\hfill
\begin{subfigure}[t]{0.47\textwidth}
\includegraphics[width=\textwidth]{/Users/bradc/Documents/University_of_Manitoba/Thesis/PhD/Chapter1/figs/m0w11.pdf}
\caption{\textit{Irregular} initial data for $\sigma=1.1$}
\label{f:m0w11}
\end{subfigure}
\caption{Classes of initial data for massless scalars and initial width
$\sigma$.  Blue dots represent horizon formation;
red triangles indicate a lower limit for $t_H$.  Red curves in subfigures
\ref{f:m0w025},\ref{f:m0w085} are $t_H=a\epsilon^{-2}+b$ matched to largest
two amplitudes in the curve.}
\label{f:classes}
\end{figure}

As a result, initial data can be divided into several classes with respect
to behavior at low amplitudes, as illustrated in figure \ref{f:classes}
for massless scalars.  \textit{Stable} initial data evolves
indefinitely without forming a horizon.  In practice, we identify this
type of behavior in numerical evolutions by noting rapid horizon formation
at high amplitude with a vertical asymptote in $t_H$ just above some
critical amplitude.  In our numerical results, we see
a sudden jump at the critical amplitude to evolutions with no horizon formation
to a large time $t_{lim}$, possibly with a small
window of amplitudes with large $t_H$ just above the critical amplitude.
In a few cases, we have captured a greater portion of the asymptotic region.
See figure \ref{f:m0w15}.
\textit{Unstable} initial data, in contrast, forms a horizon at all
amplitudes following the perturbative scaling relation $t_H\propto \epsilon^{-2}$
as $\epsilon\to 0$. In our analysis, we will verify this scaling by
fitting $t_H$ to a power law as
described in section \ref{s:methods} below; if we limit the fit to smaller
values of $\epsilon$, the scaling becomes more accurate.  Figure \ref{f:m0w025}
shows unstable data. The red curve is of the form $t_H=a\epsilon^{-2}+b$
with $a,b$ determined by matching the curve to the data for the largest
two amplitudes with $t_H\geq 60$ (not a best fit); note that the data
roughly follows this curve.
The categorization of different initial data profiles with similar
characteristic widths into stable and unstable is robust for massless and
massive scalars \cite{1508.02709}; small and large width initial data are
unstable, while intermediate widths are stable.  One of the major results
of this paper is determining how the widths of initial data in these
``islands of stability'' vary with scalar mass.

A priori, there are other possible types of behavior, at least beyond
the first subleading order in perturbation theory.
\textit{Metastable} initial data collapses with $t_H\propto\epsilon^{-p}$ with
$p>2$ at small amplitudes (or another more rapid growth of $t_H$ as
$\epsilon\to 0$).  We will find this type of behavior common on the
``shoreline'' of islands of stability where stable behavior transitions to
unstable.  As we will discuss further below, metastable
behavior may or may not continue as $\epsilon\to 0$;
in principle, as higher order terms in perturbation theory become less
important, the behavior may shift to either stable or unstable as described
above.  We in fact find circumstantial evidence in favor of the different
possibilities.
Figure \ref{f:m0w085} shows metastable initial data that continues to collapse
to times $t_H\sim 0.6t_{lim}$ but more slowly than $\epsilon^{-2}$;
note that $t_H$ for collapsed evolutions at small amplitudes lies significantly
above the curve $t_H=a\epsilon^{-2}+b$ (which is determined as in
figure \ref{f:m0w025}).
There was one additional type of behavior identified by \cite{1508.02709},
which was called ``quasi-stable'' initial data at the time
since the low-amplitude
behavior was not yet clear.  We find here that these initial data are typically
stable at small amplitude but exhibit irregular, often strongly non-monotonic
or even chaotic,
behavior in $t_H$ as a function of $\epsilon$, so we will denote them as
\textit{irregular} initial data.  Figure \ref{f:m0w11} shows an example of
irregular initial data.  Later, we will see more striking examples of
this behavior for massive scalars.


\subsection{Methods}\label{s:methods}

For spherically symmetric motion,
the Klein-Gordon equation for scalar mass $\mu$ can be written in first order
form as
\begin{align}
\phi_{,t}=&Ae^{-\delta}\Pi,\quad\Phi_{,t}=\left(Ae^{-\delta}\Pi\right)_{,x},
\label{evoleqns}\\
\Pi_{,t}=&\frac{(Ae^{-\delta}\tan^{d-1}(x)\Phi)_{,x}}{\tan^{d-1}(x)}-
\frac{e^{-\delta}\mu^2\phi}{\cos^2(x)}\ ,\label{KGequation}
\end{align}
where $\Pi$ is the canonical momentum and $\Phi=\phi_{,x}$ is an auxiliary
variable.  The Einstein equation reduces to constraints, which can be
written as
\begin{align}
  \label{deltaDeriv}
  \delta_{,x}=&-\sin(x)\cos(x)(\Pi^2+\Phi^2)\\
  \label{massDeriv}
  M_{,x}=&\left(\tan(x)\right)^{d-1}\left[
    A\frac{\left(\Pi^2+\Phi^2\right)}{2}+
    \frac{\mu^2\phi^2}{2\cos^2(x)}\right],\\
  A=&1-2\frac{\sin^2(x)}{(d-1)}\frac{M}{\tan^{d}(x)},
\end{align}
where the mass function $M$ asymptotes to the conserved ADM mass at the
boundary $x=\pi/2$.  We will restrict to $d=4$ spatial dimensions.
Since results are robust against changes in the type of
initial data \cite{1508.02709},
we can take the initial data to be a Gaussian of the areal radius in the
canonical momentum and trivial in the field.  Specifically,
\beq{PiGaussianID}
\Pi(t=0,x)=\epsilon\exp\left(-\frac{\tan^2(x)}{\sigma^2}\right),
\quad\phi(t=0,x)=0.
\eeq
The width $\sigma$ and field mass $\mu$ constitute the parameter space
for our phase diagram.

We solve the Klein-Gordon evolution equations (\ref{evoleqns},\ref{KGequation})
and Einstein constraint equations (\ref{deltaDeriv},\ref{massDeriv})
numerically using methods similar to those of \cite{1308.1235} on a spatial
grid of $2^n+1$ grid points; we discuss the
convergence properties of our code in the appendix.
We denote the approximate horizon position $x_H$ and formation time $t_H$
by the first point such that $A(x_H,t_H)\leq 2^{7-n}$.
In detail, we evolve the system in time using a 4th-order Runge-Kutta
stepper and initially use a 4th-order Runge-Kutta spatial integrator at
resolution $n=14$.  If necessary, we switch to a 5th-order Dormand-Prince
spatial integrator and increase resolution near horizon formation.  Due to
time constraints, we do not increase the resolution beyond $n=21$ for any
particular calculation; if a higher resolution would be required to track
horizon formation for a given amplitude, we exclude that amplitude.

To determine the stability class of initial data with a given width $\sigma$,
we allow evolutions to run to a maximum time of $t_{lim}=500$ in AdS units, so
$t_{lim}$ is a lower limit for $t_H$ for amplitudes that do not form a horizon
within that time.  Normally, however, if the initial data appears unstable,
we only evolve amplitudes with $t_H\lesssim 0.6t_{lim}$; this is partly to
save computational resources and partly to distinguish stable evolutions from
collapsing ones.  For unstable or metastable initial data, we find the
best fit of the form $t_H=a \epsilon^{-p}+b$ to evolutions with $t_H>t_{fit}$,
where $t_{fit}$ is a constant time chosen such that amplitudes with
evolutions that last longer are usually roughly perturbative;
in practice, $t_{fit}=60$ gives results close to the perturbative result
$p=2$ for evolutions expected to be unstable from the literature, but we will
also consider $t_{fit}=80,100$ as described below.


\section{Phases}\label{s:phases}
Here we give our main result, the phase diagram of stability classes
as a function of initial profile width and scalar mass, along with a
more detailed discussion of the scaling of horizon formation time with
amplitude for varying initial data.

\begin{figure}[!t]
\centering
\includegraphics[width=\textwidth]{/Users/bradc/Documents/University_of_Manitoba/Thesis/PhD/Chapter1/figs/phasediag.pdf}
\caption{Phase diagram as a function of initial data width $\sigma$ and
scalar mass $\mu$.  Filled circles represent the unstable phase, empty circles
the stable phase, top-half-filled circles metastability, and right-half-filled
circles the irregular phase.}
\label{f:phase}
\end{figure}

The stability phase diagram for spherically
symmetric scalar field collapse in AdS$_5$, treating the width $\sigma$
of initial data and scalar field mass $\mu$ as tunable parameters,
appears in figure \ref{f:phase}.  Each $(\mu,\sigma)$ combination that
we evolved numerically is indicated by a circle, with filled and empty
circles representing the unstable and stable phases respectively.  The
metastable phase is represented by circles filled in the top half, while
the irregular phase are filled in the right half.  At a glance, two features
of the phase diagram are apparent: as $\mu$ increases, the island
of stability moves toward smaller values of $\sigma$ and takes up a
gradually larger range of $\sigma$.  To be specific, the stable phase
is centered at $\sigma=\bar\sigma\sim 1.4$ and has a width of
$\Delta\sigma\sim 0.7$ for $\mu=0,0.5$, with $\bar\sigma\sim 1.2$ for $\mu=1$.
$\Delta\sigma$ increases to $\sim 1.1$, and the island of
stability is centered at
$\bar\sigma\sim 0.9$ for $\mu=5,10$, while $\Delta\sigma\sim 1.2$
for $\mu=15,20$ with the stable phase centered at $\bar\sigma\sim 0.8$.
Note that the transition between ``light field'' and ``heavy field'' 
behavior occurs for $\mu>1$ in AdS units.

The metastable and irregular phases appear at the shorelines of the island
of stability, the boundary between unstable and stable phases.  In particular,
the slope of the power law $t_H\sim \epsilon^{-p}$ as $\epsilon\to 0$
increases as the width moves toward the island of stability, leading to a
metastable phase.  We find metastability at the large $\sigma$ shoreline
for all $\mu$ values considered and also at the small $\sigma$ shoreline
for several scalar masses.  It seems likely that metastable behavior appears
in only a narrow range of $\sigma$ for larger $\mu$, which makes it harder
to detect in a numerical search, leading to its absence in some parts of the
phase diagram.  We find the irregular phase at the small $\sigma$ shoreline
for every mass and at the large $\sigma$ boundary for large $\mu$,
closer to stable values of $\sigma$ than the metastable phase.  This phase
includes a variety of irregular and non-monotonic behavior, as detailed below.
Truly chaotic behavior especially becomes more prominent at larger values of
$\mu$, as we will discuss below.

\subsection{Metastable versus unstable phases}\label{s:metastable}

While the stable and irregular phases are typically apparent by eye in a plot
of $t_H$ vs $\epsilon$, distinguishing the unstable from metastable phase
is a quantitative task.  As we described in section \ref{s:methods}, we
find the least squares fit of $t_H=a\epsilon^{-p}+b$ to all evolutions
with $t_H>t_{fit}$ for the given $(\mu,\sigma)$, running over values
$t_{fit}=60,80,100$.  Using the covariance matrix of the fit, we also find
the standard error for each fit parameter.  We classify a width as having
unstable evolution if the best fit value of $p$ is within two standard
errors of $p=2$ for $t_{fit}=60,80$ or one standard error for $t_{fit}=100$
(due to a smaller number of data points, the standard errors for $t_{fit}=100$
tend to be considerably larger).\footnote{Except for poor fits as described
in our discussion of the irregular phase.} Considering larger values of $t_{fit}$
helps to ensure that the particular initial profile does not reach the
perturbative regime at smaller amplitude values.

\begin{figure}[t]
\centering
\includegraphics[width=0.5\textwidth]{/Users/bradc/Documents/University_of_Manitoba/Thesis/PhD/Chapter1/figs/CollapseCoefficientvsWidth.pdf}
\caption{Coefficient $a$ from the fit $t_H=a\epsilon^{-p}+b$ as a function
of width $\sigma$ using $t_{fit}=60$. 
Shows data for $\mu=0$ (green diamonds), $0.5$ 
(red triangles), $1$ (yellow stars), 5 (black circles), 10 (cyan squares),
15 (magenta Y), and 20 (blue circles).  The orange line is the best 
power law fit.}
\label{f:unstable}
\end{figure}

The fits $t_H=a\epsilon^{-p}+b$ allow us to explore the time scale of 
horizon formation across the phase diagram, for example through a contour
plot of one of the coefficients vs $\sigma$ and $\mu$.  In most cases,
this has not been informative, but an intriguing feature emerges if we
plot the normalization coefficient $a$ vs $\sigma$ for unstable initial 
data at small $\sigma$, as shown in figure \ref{f:unstable} for $t_{fit}=60$.  
By eye, the coefficient is reasonably well described by the fit
$a=32.0(3) \sigma^{-2.01(2)}$ (values in parentheses are standard errors of
the best fit values) \textit{independent of scalar field mass}.
This is not born out very well quantitatively; the reduced $\chi^2$ for the
fit is $\chi^2$/d.o.f.$=180$, indicating a poor fit.  However, the large
$\chi^2$ seems largely driven by a few outlier points with large
scalar mass, so it is tempting
to speculate that the gravitational collapse in this region of parameter
space is driven by gradient energy, making all fields effectively massless
at narrow enough initial $\sigma$.  The picture is qualitatively similar
if we consider the parameter $a$ for $t_{fit}=80,100$ instead.


\begin{figure}[!t]
\centering
\begin{subfigure}[t]{0.47\textwidth}
\includegraphics[width=\textwidth]{/Users/bradc/Documents/University_of_Manitoba/Thesis/PhD/Chapter1/figs/m15w150fit.pdf}
\caption{$\mu=15,\sigma=1.5$}
\label{f:m15w150}
\end{subfigure}\hfill
\begin{subfigure}[t]{0.47\textwidth}
\includegraphics[width=\textwidth]{/Users/bradc/Documents/University_of_Manitoba/Thesis/PhD/Chapter1/figs/m5w170fit.pdf}
\caption{$\mu=5,\sigma=1.7$}
\label{f:m5w170}
\end{subfigure}
\begin{subfigure}[t]{0.47\textwidth}
\includegraphics[width=\textwidth]{/Users/bradc/Documents/University_of_Manitoba/Thesis/PhD/Chapter1/figs/m0w18fit.pdf}
\caption{$\mu=0,\sigma=1.8$}
\label{f:m0w18}
\end{subfigure}\hfill
\begin{subfigure}[t]{0.47\textwidth}
\includegraphics[width=\textwidth]{/Users/bradc/Documents/University_of_Manitoba/Thesis/PhD/Chapter1/figs/m05w17fit.pdf}
\caption{$\mu=0.5,\sigma=1.7$}
\label{f:m05w17}
\end{subfigure}
\caption{Metastable behavior: blue dots represent horizon formation and
red triangles a lower limit on $t_H$.  Magenta curves are fits
$t_H=a\epsilon^{-p}+b$ over the shown range of amplitudes. See table
\ref{t:figfits} for best fit parameters.}
\label{f:metastable}
\end{figure}

Several examples of metastable behavior appear in figure \ref{f:metastable}.
These figures show both data from the numerical evolutions (blue dots and
red triangles) and fits of the form $t_H=a\epsilon^{-p}+b$ for points with
$t_H>t_{fit}=60$ (magenta curves).  The best fit parameters are given in
table \ref{t:figfits}.

Figures \ref{f:m15w150},\ref{f:m5w170} demonstrate behavior typical of most
of the instances of metastable initial data we have found; specifically,
the initial data continue to collapse through horizon formation times of
$t_H\sim 0.6 t_{lim}$ but with $p$ significantly greater than the perturbative
value of $p=2$. Note that the evolutions of figure \ref{f:m5w170} have
been extended to larger
values of $t_H$ to demonstrate that the evolutions continue to collapse to
somewhat smaller amplitude values.  Figure \ref{f:m5w170} is also of
interest because its best fit value $p\approx 2.08$ is approximately as close
to the perturbative value as several stable sets of initial data but has a
smaller standard error for the fit, so the difference from the perturbative
value is more significant.

\begin{table}[!t]
\begin{center}
%\begin{footnotesize}
\begin{tabular}{|c|c|c|c|c|}
\hline
&$a$&$p$ &$b$ & $\chi^2/$d.o.f.\\
\hline
$\mu=15,\sigma=1.5$& 0.10(1)&2.33(5)& -27(4)&0.7736\\
\hline
$\mu=5,\sigma=1.7$&0.89(5) &2.08(2)& -33(2)&0.3248\\
\hline
$\mu=0,\sigma=1.8$&0.06(2)&4.3(2)& 30(5)& 1.502\\
\hline
$\mu=0.5,\sigma=1.7$ ($t_H<0.4t_{lim}$)&4(32)$\times 10^{-45}$ &73(5) &70(2)
&5.409 \\
($t_H>0.72t_{lim}$)& 0.02(3) &5.6(8) &260(20) &1.078 \\
\hline
\end{tabular}
%\end{footnotesize}
\end{center}
\caption{Best fit parameters for the cases shown in figure \ref{f:metastable}
restricting to $t_H>t_{fit}=60$ and as noted.  Values in parentheses are
standard errors in the last digit. $\chi^2/$d.o.f. is the
reduced $\chi^2$ value used as a measure of goodness-of-fit. }
\label{t:figfits}
\end{table}

Figure \ref{f:m0w18} shows metastable evolution to $t_H\lesssim 0.6t_{lim}$
but then a sudden jump to stability until $t=t_{lim}$.  In the figure, the
fit has been extended to the largest non-collapsing amplitude,
which demonstrates that
there is no collapse over a time period significantly longer than the fit
predicts. This example argues that metastable data
may in fact become stable at the smallest amplitudes.  On the other hand,
figure \ref{f:m05w17} shows a similar jump in $t_H$ to values $t_H<t_{lim}$;
evolution at lower amplitudes shows metastable scaling with $p\approx 6$ 
for $360<t_H<t_{lim}$.  The figure also shows a metastable fit 
with larger reduced $\chi^2$ at
larger amplitudes corresponding to $t_{fit}<t_H<0.4 t_{lim}$.
So this is another option: metastable behavior
may transition abruptly to metastable behavior with different scaling
(or possibly even perturbatively unstable behavior) at sufficiently
small amplitudes. It is also reasonable to classify this case as irregular
due to the sudden jump in $t_H$; we choose metastable due to the clean 
metastable behavior at low amplitudes.

%\textbf{PLOT AND DISCUSSION OF POWER AS FUNCTION OF WIDTH??}


\subsection{Behaviors of the irregular phase}

We have found a variety of irregular behaviors at the transition
between the metastable and stable phases which we have classified together
as the irregular phase; however, it may be better to describe them as
separate phases.  The phase diagram \ref{f:phase} indicates that the
irregular phase extends along the ``inland'' side of the small $\sigma$
shoreline and at least part of the large $\sigma$ shoreline
of the island of stability.  What is not clear from our evolutions up to
now is whether each type of behavior appears along the entire shoreline
or if they appear in pockets at different scalar field masses. Examples
of each type of behavior that we have found appear in figure \ref{f:irregular}.

\begin{figure}[!t]
\centering
\begin{subfigure}[t]{0.47\textwidth}
\includegraphics[width=\textwidth]{/Users/bradc/Documents/University_of_Manitoba/Thesis/PhD/Chapter1/figs/m05w1.pdf}
\caption{$\mu=0.5,\sigma=1$}
\label{f:m05w1}
\end{subfigure}\hfill
\begin{subfigure}[t]{0.47\textwidth}
\includegraphics[width=\textwidth]{/Users/bradc/Documents/University_of_Manitoba/Thesis/PhD/Chapter1/figs/m5w034.pdf}
\caption{$\mu=5,\sigma=0.34$}
\label{f:m5w034}
\end{subfigure}
\begin{subfigure}[t]{0.47\textwidth}
\includegraphics[width=\textwidth]{/Users/bradc/Documents/University_of_Manitoba/Thesis/PhD/Chapter1/figs/m20w016.pdf}
\caption{$\mu=20,\sigma=0.16$}
\label{f:m20w016}
\end{subfigure}\hfill
\begin{subfigure}[t]{0.47\textwidth}
\includegraphics[width=\textwidth]{/Users/bradc/Documents/University_of_Manitoba/Thesis/PhD/Chapter1/figs/m20w019.pdf}
\caption{$\mu=20,\sigma=0.19$}
\label{f:m20w019}
\end{subfigure}
\caption{Irregular behavior: blue dots represent horizon formation and
red triangles a lower limit on $t_H$.
}
\label{f:irregular}
\end{figure}

The first type of irregular behavior, shown in figure \ref{f:m05w1}, is
monotonic ($t_H$ increases with decreasing $\epsilon$ as usual), but it is
not well fit by a power law.  In fact, this behavior would classify as
metastable by the criterion of section \ref{s:metastable} in that the
power law of the best fit $t_H=a\epsilon^{-p}+b$ is significantly different
from $p=2$, except for the fact that the reduced $\chi^2$ value for the
fit is very large (greater than 10)
and also that different fitting algorithms can return
significantly different fits, even though the data may appear to the eye
like a smooth power law.  In any case, this type of behavior apparently
indicates a breakdown of metastable behavior and hints at the appearance of
non-monotonicity.  So far, our evolutions have not demonstrated sudden jumps
in $t_H$ typical of stability at low amplitudes, however.

Figure \ref{f:m5w034} exemplifies non-monotonic behavior in the irregular
phase.  This type of behavior, which was noted already by \cite{1304.4166},
involves one or more sudden jumps in $t_H$ as $\epsilon$ decreases,
which may be followed by a sudden decrease in $t_H$ and then resumed smooth
monotonic increase in $t_H$.  There are suggestions that this type of
initial data is stable at low amplitudes due to the usual appearance of
non-collapsing evolutions, but it is worth noting that these amplitudes could
instead experience another jump and decrease in $t_H$, just at $t_H>t_{lim}$.
Finally, \cite{1508.02709} studied this type of behavior in some detail,
denoting it as ``quasi-stable.''

The last type of irregular behavior is apparently chaotic, in that $t_H$
appears to be sensitive to initial conditions (ie, value of amplitude) over
some range of amplitudes. This type of behavior appears over the range
of masses (see figure \ref{f:m0w11} for a mild case for massless scalars),
but it
is more common and more dramatic at larger $\mu$. Figures
\ref{f:m20w016},\ref{f:m20w019} represent the most extreme chaotic behavior
among the initial data that we studied with collapse at $t_H<50$ not very
far separated from amplitudes that do not collapse for $t<t_{lim}$ along with
an unpredictable pattern of variation in $t_H$.  This type of chaotic behavior
has been seen previously in the collapse of transparent but gravitationally
interacting thin shells in AdS \cite{Brito:2016xvw} as well as in the
collapse of massless scalars in AdS$_5$ Einstein-Gauss-Bonnet gravity
\cite{1410.1869,1608.05402}.  In both cases, the chaotic behavior is
hypothesized to be due to the transfer of energy between two infalling shells,
with horizon formation only proceeding when one shell is sufficiently
energetic.  In the latter case, the extra scale of the theory
(given by the coefficient of the Gauss-Bonnet term in the action) leads the
single initial pulse of scalar matter to break into two pulses.

We should therefore ask two questions: does this irregular behavior show
evidence of true chaos, and is a similar mechanism at work here? To quantify
the presence of chaos, we examine the difference in time evolution between
similar initial conditions (nearby amplitudes), which diverge exponentially
in chaotic systems. Specifically, any quantity $\Delta$ should satisfy
$|\Delta| \propto \exp(\lambda t)$ for Lyapunov coefficient $\lambda$.
Our characteristic will be the
upper envelope of the Ricci scalar at the origin per light crossing time,
$\bar{\mathcal R}(t)$. We consider the chaotic behavior exhibited by three
states: a massless scalar of width $\sigma = 1.1$ with amplitudes
$\epsilon=1.02,1.01,1.00$ (see figure \ref{f:m0w11}),
a $\mu = 5$ massive scalar of width $\sigma = 0.34$ and
$\epsilon=3.52,3.51,3.50$, and a $\mu = 20$ scalar of width $\sigma = 0.19$
and $\epsilon=6.98,6.95,6.92$ (figure \ref{f:m20w019}).

\begin{figure}[!t]
\centering
\begin{subfigure}[t]{0.47\textwidth}
\includegraphics[width=\textwidth]{/Users/bradc/Documents/University_of_Manitoba/Thesis/PhD/Chapter1/figs/m5Ricci}
\caption{Upper envelope of Ricci scalar at origin}
\label{f:m5Ricci}
\end{subfigure}\hfill
\begin{subfigure}[t]{0.47\textwidth}
\includegraphics[width=\textwidth]{/Users/bradc/Documents/University_of_Manitoba/Thesis/PhD/Chapter1/figs/m5Lyapunov}
\caption{$\log | \Delta |$ vs. $t_{mid}$}
\label{f:m5Lyapunov}
\end{subfigure}
\caption{
Left: The upper envelope of the Ricci scalar for amplitudes
$\epsilon_1=3.50$ (blue circles), $\epsilon_2=3.51$ (red triangles), and
$\epsilon_3=3.52$ (green squares) for $\mu=5,\sigma=0.34$.
Right: $\log(|\Delta_{12}|)$ and best fit (blue
circles and line) and $\log(|\Delta_{23}|)$ and best fit (red squares and line),
calculated as a function of the midpoint $t_{mid}$ of the time interval.}
\label{f:m5chaotic}
\end{figure}

Figure \ref{f:m5chaotic} details evidence for chaotic evolution in the
$\mu=5,\sigma=0.34$ case; figure \ref{f:m5Ricci} shows our characteristic
function $\bar{\mathcal{R}}(t)$ for the amplitudes $\epsilon_1 = 3.50$,
$\epsilon_2 = 3.51$, and $\epsilon_3 = 3.52$. By eye, $\bar{\mathcal{R}}$ shows
noticeable differences after a long period of evolution. These are more
apparent in figure \ref{f:m5Lyapunov}, which shows the log of the differences
$\Delta_{ab}\equiv\bar{\mathcal{R}}_{\epsilon_a}-\bar{\mathcal R}_{\epsilon_b}$,
along with the best fits. Although there is considerable noise --- or
oscillation around exponential growth --- in the
differences (leading to $R^2$ values $\sim 0.2,0.26$ for the fits), the
average slope gives Lyapunov coefficient $\lambda=0.007$ (within the error
bar of each slope), and each slope differs from zero by more than 3 standard
errors. One interesting point is that the $t_H$ vs $\epsilon$ curve
in figure \ref{f:m5w034} does not appear chaotic to the eye, even though it
shows some of the mathematical signatures of chaos at least for
$\epsilon_1<\epsilon<\epsilon_3$, the amplitude values near the visible spike
in $t_H$.

The story is similar for the massless and $\mu=20$ cases we
studied, which exhibit $\lambda$ values that
differ from zero by at least 1.9 standard deviations; see table \ref{t:lyap}.
This is a milder version of the chaotic behavior noted by
\cite{Brito:2016xvw,1410.1869,1608.05402}, especially for the $\mu=5$ case
studied.  To our knowledge, this is the first evidence of chaos in the
gravitational collapse of a massless scalar in AdS.
One thing to note is that the strength of oscillation in
$\log(|\Delta|)$ around the linear fit increases with increasing
mass, so that the two best fit Lyapunov exponents for $\mu = 20$ are no
longer consistent with each other at the 1-standard deviation level.


\begin{table}[!t]
\begin{center}
\begin{tabular}{|cc|c|c|}
\hline
&&$\lambda$ &average $\lambda$\\
\hline
$\mu=0,\sigma=1.1$ & $\Delta_{12}$ &0.011(6) & 0.011\\
&$\Delta_{23}$ & 0.011(5) & \\
\hline
$\mu=5,\sigma=0.34$ & $\Delta_{12}$ &0.006(2)& 0.007\\
&$\Delta_{23}$ & 0.007(2)& \\
\hline
$\mu=20,\sigma=0.19$ & $\Delta_{12}$ &0.046(9)& 0.032\\
&$\Delta_{23}$ & 0.019(7) & \\
\hline
\end{tabular}
\end{center}
\caption{Best fit Lyapunov coefficients $\lambda$ for adjacent amplitude
pairs and average $\lambda$ value for each $\mu,\sigma$ system studied.
The parenthetical value is the error in the last digit.}
\label{t:lyap}
\end{table}

The mechanism underlying the chaotic behavior seems somewhat different
or at least weaker than the two-shell or Einstein-Gauss-Bonnet systems.
When examining the time evolution of the mass distributions of
these data, we see a single large pulse of mass energy that oscillates
between the origin and boundary without developing a pronounced peak.
However, there is also apparently a smaller wave that travels across the
large peak.  In the massless case examined, this wave deforms the pulse,
leading at times to the double-shoulder appearance of figure \ref{f:m0hump}.
In the $\mu=5,\sigma=0.34$ case, the secondary wave is more like a ripple,
usually smaller in amplitude but more sharply localized, as toward the right
side of the main pulse in figure \ref{f:m5ripple}.  So the chaotic behavior
may be caused by the relative motion of the two waves, rather than energy
transfer between two shells.  In this hypothesis, a horizon would form when
both waves reach the neighborhood of the origin at the same time.

\begin{figure}[!t]
\centering
\begin{subfigure}[t]{0.47\textwidth}
\includegraphics[width=\textwidth]{/Users/bradc/Documents/University_of_Manitoba/Thesis/PhD/Chapter1/figs/m0chaoshump.pdf}
\caption{$\mu=0,\sigma=1.1,\epsilon=1.01,t=60$}
\label{f:m0hump}
\end{subfigure}
\begin{subfigure}[t]{0.47\textwidth}
\includegraphics[width=\textwidth]{/Users/bradc/Documents/University_of_Manitoba/Thesis/PhD/Chapter1/figs/m5chaosripple.pdf}
\caption{$\mu=5,\sigma=0.34,\epsilon=3.52,t=137$}
\label{f:m5ripple}
\end{subfigure}
\caption{Radial derivative of the mass function at the indicated time
for two chaotic systems.  Note the appearance of a secondary wave on top
of the main pulse. $(\mu,\sigma,\epsilon)$ as indicated.}
\label{f:chaosmechanism}
\end{figure}

As a note, we have run convergence tests on several sets of irregular
initial data and find that our calculations are convergent overall, as expected
(even at lower resolution than we used).  In particular, the massless
scalar evolutions studied in table \ref{t:lyap} are convergent already at
resolution given by $n=12$ (note that we typically start at $n=14$).
The only caveat may be for some of the strongly chaotic initial data with 
scalar mass $\mu=20$, which nonetheless appear well-behaved according to other 
indicators.  The reader may or may not wish to take them at
face value but should recall that we have presented other chaotic initial
data with rigorously convergent evolutions.  See the appendix for a more 
detailed discussion.


\section{Spectral analysis}\label{s:analysis}

As we discussed in the introduction, instability toward horizon formation
proceeds through a turbulent cascade of energy to shorter wavelengths or,
more quantitatively, to 1st-order scalar eigenmodes with more nodes.
Inverse cascades are typical of stable evolutions.
Therefore, understanding the energy spectrum of our evolutions, both initially
and over time, sheds light on the behavior of the self-gravitating scalar
field in asymptotically AdS spacetime, providing a heuristic analytic 
understanding of the phase diagram.

The (normalizable) eigenmodes $e_j$ are given by Jacobi polynomials as
\beq{emodes}
 e_j(x)=\kappa_j\cos^{\lambda_+}(x) P^{(d/2-1,\sqrt{d^2+4\mu^2}/2)}_j(\cos(2x))
\eeq
($\kappa_j$ is a normalization constant)
with eigenfrequency $\omega_j=2j+\lambda_+$ and
$\lambda_+=(d+\sqrt{d^2+4\mu^2})/2$ in AdS$_{d+1}$ for $j=0,1,\cdots$
(see \cite{hep-th/9905111,Nastase:2015wjb} for reviews).
Including gravitational backreaction, we define the energy spectrum
\beq{spectrum}
E_j\equiv\frac{1}{2}\left(\Pi_j{}^2-\phi_j\ddot{\phi}_j\right),\eeq
where
\begin{eqnarray}
\Pi_j&=&\left(\sqrt{A}\Pi,e_j\right),\quad
\phi_j=\left(\phi,e_j\right),\nonumber\\
\ddot\phi_j&=&\left(\cot^{d-1}(x)\partial_x\left[\tan^{d-1}(x)A\Phi\right]
-\mu^2\sec^2(x)\phi,e_j\right),
\end{eqnarray}
and the inner product is $(f,g)=\int_0^{\pi/2}dx\tan^{d-1}(x) fg$.
The sum of $E_j$ over all modes is the conserved ADM mass.

\subsection{Dependence on mass}
\begin{figure}[!t]
\centering
\begin{subfigure}[t]{0.45\textwidth}
\includegraphics[width=\textwidth]{/Users/bradc/Documents/University_of_Manitoba/Thesis/PhD/Chapter1/figs/gaussianspectra.pdf}
\caption{Best fit gaussian energy spectra.} 
\label{f:gaussian}
\end{subfigure} \hfill
\begin{subfigure}[t]{0.45\textwidth}
\includegraphics[width=\textwidth]{/Users/bradc/Documents/University_of_Manitoba/Thesis/PhD/Chapter1/figs/gaussianfitting.pdf}
\caption{Best fit gaussian and zeroth eigenmode.}
\label{f:gaussianfits}
\end{subfigure}
\caption{Left: Spectra of the best fit gaussians (\ref{PiGaussianID}) to 
the $j = 0$ eigenmode for masses $\mu=0$ (blue circles), 0.5 (yellow squares),
1 (empty orange circles), 5 (green diamonds), 10 (empty cyan squares), 
15 (upward red triangles), and 20 (downward purple
triangles). Right: an overlay of the best fit Gaussian and $e_0$ eigenmode 
for $\mu = 0$ (solid blue is best fit, orange dashed is eigenmode) and $\mu = 20$ (solid green, red short dashes).}
\label{f:gaussians}
\end{figure}

The most visibly apparent feature of the phase diagram of figure \ref{f:phase}
is that the island of stability both expands and shifts to smaller widths
as the scalar mass increases.  As it turns out, the energy spectrum of the
Gaussian initial data (\ref{PiGaussianID}) provides a simple heuristic
explanation.

It is well established both in perturbation theory and numerical studies that
initial data given by a single scalar linear-order eigenmode is in fact
nonlinearly stable, and the spectra of many quasi-periodic solutions are also
dominated by a single eigenmode.  As a result, we should expect Gaussian
initial data that approximates a single eigenmode (which must be $j=0$ due to
lack of nodes) to be stable.  To explore how this depends on mass, we find the
best fit values of $\epsilon,\sigma$ for the $j=0$ eigenmode for each mass
that we consider (defined by the least-square error from the Gaussian to a
discretized eigenmode);
this is the ``best approximation'' Gaussian to the eigenmode.
Then we find the energy spectrum of that best-fit Gaussian; these are shown
in figure \ref{f:gaussian}.  From the figure, it is clear that the $j=0$
eigenmode is closer to a Gaussian at larger masses.  That is, other
eigenmodes contribute less to the Gaussian's spectrum at higher masses (by
several orders of magnitude over the range from $\mu=0$ to 20).  Simply put,
the shape of the $j=0$ eigenmode is closer to Gaussian at higher masses,
which suggests that the island of stability should be larger at larger
scalar field mass. Figure \ref{f:gaussianfits} compares the $j=0$ eigenmode
and best fit Gaussian for $\mu=0$ and 20; on inspection, there is more
deviation between the eigenmode and Gaussian for the massless scalar.

In addition, the best-fit Gaussian width decreases from $\sigma\sim 0.8$
for a massless scalar as the mass increases.  At $\mu=20$, the best-fit
width is $\sigma\sim 0.31$. This suggests that Gaussians that approximate the
$j=0$ mode well enough are narrower in width at higher masses.  An interesting
point to note is that the island of stability for $\mu=0,0.5$ is actually
centered at considerably larger widths than the best-fit Gaussian.  This may
not be surprising, since the best-fit Gaussians at low masses actually
receive non-negligible contributions from higher mode numbers; moving away from
the best-fit Gaussian can actually reduce the power in higher modes.
For example, the stable initial data shown in figure \ref{f:m0w15spect} below
has considerably less power in the $j=2$ mode.





\subsection{Spectra of different phases}
\begin{figure}[!t]
\centering
\begin{subfigure}[t]{0.32\textwidth}
\includegraphics[width=\textwidth]{/Users/bradc/Documents/University_of_Manitoba/Thesis/PhD/Chapter1/figs/m0w15spect.pdf}
\caption{$\mu=0,\sigma=1.5$}
\label{f:m0w15spect}
\end{subfigure}
\begin{subfigure}[t]{0.32\textwidth}
\includegraphics[width=\textwidth]{/Users/bradc/Documents/University_of_Manitoba/Thesis/PhD/Chapter1/figs/m0w025spect.pdf}
\caption{$\mu=0,\sigma=0.25$}
\label{f:m0w025spect}
\end{subfigure}
\begin{subfigure}[t]{0.32\textwidth}
\includegraphics[width=\textwidth]{/Users/bradc/Documents/University_of_Manitoba/Thesis/PhD/Chapter1/figs/m5w170spect.pdf}
\caption{$\mu=5,\sigma=1.7$}
\label{f:m5w170spect}
\end{subfigure}
\begin{subfigure}[t]{0.32\textwidth}
\includegraphics[width=\textwidth]{/Users/bradc/Documents/University_of_Manitoba/Thesis/PhD/Chapter1/figs/m05w1spect.pdf}
\caption{$\mu=0.5,\sigma=1$}
\label{f:m05w1spect}
\end{subfigure}
\begin{subfigure}[t]{0.32\textwidth}
\includegraphics[width=\textwidth]{/Users/bradc/Documents/University_of_Manitoba/Thesis/PhD/Chapter1/figs/m5w034spect.pdf}
\caption{$\mu=5,\sigma=0.34$}
\label{f:m5w034spect}
\end{subfigure}
\begin{subfigure}[t]{0.32\textwidth}
\includegraphics[width=\textwidth]{/Users/bradc/Documents/University_of_Manitoba/Thesis/PhD/Chapter1/figs/m20w016spect.pdf}
\caption{$\mu=20,\sigma=0.16$}
\label{f:m15w150spect}
\end{subfigure}
\caption{Initial ($t=0$) energy spectra for the indicated evolutions. In order,
these represent stable, unstable, metastable, monotonic irregular,
non-monotonic irregular, and chaotic irregular initial data.}
\label{f:initialspect}
\end{figure}

A key question that one might hope to answer is whether the stability phase
of a given $(\mu,\sigma)$ can be determined easily by direct inspection
of the initial data without requiring many evolutions at varying amplitudes.
The initial energy spectra for examples of each phase, including monotonic,
non-monotonic, and chaotic irregular behaviors, are shown in figure
\ref{f:initialspect}.  These spectra are taken from among the smallest
amplitudes we evolved in order to minimize backreaction effects.

Unfortunately, the initial energy spectra do not seem to provide such a
method for determining the stability phase.  Very broadly speaking, stable
and metastable $(\mu,\sigma)$ correspond to initial spectra that drop off
fairly quickly from the $j=0$ mode as $j$ increases, while unstable and
irregular phases tend to have roughly constant or even slightly increasing
spectra up to $j=5$ or $10$.  However, figure \ref{f:m05w1spect} shows that
some irregular initial data have spectra that decrease rapidly after a small
increase from $j=1$ to $j=2$.  Kinks in the spectrum are more prevalent for
widths of the AdS scale or larger, while spectra for smaller widths tend
to be smoother.




\subsection{Evolution of spectra}
\begin{figure}[!t]
\centering
\begin{subfigure}[t]{0.47\textwidth}
\includegraphics[width=\textwidth]{/Users/bradc/Documents/University_of_Manitoba/Thesis/PhD/Chapter1/figs/m0w18EnergyDecomp.pdf}
\caption{$\mu=0,\sigma=1.8,\epsilon=0.13$}
\label{f:m0w18decomp}
\end{subfigure}
\begin{subfigure}[t]{0.47\textwidth}
\includegraphics[width=\textwidth]{/Users/bradc/Documents/University_of_Manitoba/Thesis/PhD/Chapter1/figs/m0w025EnergyDecomp.pdf}
\caption{$\mu=0,\sigma=0.25,\epsilon=2.28$}
\label{f:m0w025decomp}
\end{subfigure}
\begin{subfigure}[t]{0.47\textwidth}
\includegraphics[width=\textwidth]{/Users/bradc/Documents/University_of_Manitoba/Thesis/PhD/Chapter1/figs/m05w17EnergyDecomp.pdf}
\caption{$\mu=0.5,\sigma=1.7,\epsilon=0.216$}
\label{f:m05w17decomp}
\end{subfigure}
\begin{subfigure}[t]{0.47\textwidth}
\includegraphics[width=\textwidth]{/Users/bradc/Documents/University_of_Manitoba/Thesis/PhD/Chapter1/figs/m20w019EnergyDecomp.pdf}
\caption{$\mu=20,\sigma=0.19,\epsilon=6.95$}
\label{f:m20w019decomp}
\end{subfigure}
\caption{The time dependence of the energy spectra as a fraction of the
total ADM mass for the indicated
$\mu,\sigma,\epsilon$.  Lower panels show the lowest 7 modes (in colors
cyan, red, purple, green, yellow, brown, and gray
respectively).  Upper panels show cumulative energy to mode
$j=0,1,2,4,8,16,32$ (in colors blue, orange, brown, yellow, aqua, red, and
magenta). }
\label{f:evolvingspectra}
\end{figure}

While the initial spectrum for a given $(\mu,\sigma)$ pair does not have
predictive value regarding the future behavior as far as
we can tell, the time dependence of the spectrum throughout the evolution of
the system is informative.  Figure \ref{f:evolvingspectra} shows the
time-dependence of spectra for examples of the stable, unstable, metastable,
and chaotic irregular phases. In each figure, the lower panel shows the
fraction $E_j/M_{ADM}$ in each mode up to $j=6$, while the upper panel shows
the cumulative fraction $\sum_j E_j/M_{ADM}$ to the mode $2^k$ with $k=0$ to 5.

The difference between stable evolution in figure \ref{f:m0w18decomp} and
unstable evolution in figure \ref{f:m0w025decomp} is readily apparent.
As the evolution proceeds, we expect a cascade of energy into higher mode
numbers, but inverse cascades to lower modes can also occur.  The stable
evolution shows a slow pattern of cascades and inverse cascades, in fact.
On the other hand, the unstable evolution shows a nearly monotonic cascade of
energy into the highest modes along with a simultaneous cascade of energy
into the lowest modes (therefore depleting intermediate modes).  These
are common observations in the literature and are included here for
completeness.

The metastable evolution shown in figure \ref{f:m05w17decomp} is interesting
in light of the stable and unstable spectra.  The amplitude shown is
from the ``unstable'' portion of figure \ref{f:m05w17}, the part consistent
with the perturbative scaling $t_H\sim\epsilon^{-2}$.  However, the spectrum
shows a similar pattern of slow cascades and inverse cascades to the
stable phase example, though on a somewhat faster time scale in this case.
While perhaps surprising, this is in keeping with the similarities noted
between the initial spectra in figures \ref{f:m0w15spect} and
\ref{f:m5w170spect}.  We have also checked that the time-dependent spectrum
at a higher amplitude with $t_H\sim 100$ follows the same pattern as
\ref{f:m05w17decomp}; in fact, it looks essentially the same but simply
ends at an earlier time.
This lends some credence to the idea that the metastable
phase is stable at lowest nontrivial order in perturbation theory, with
instability triggered by higher-order corrections.  Alternately, the
instability could be caused by an oscillatory singularity in the perturbative
theory, as discussed in \cite{1506.03519,1508.04943,1606.02712,1607.08094}
in the case of two-mode initial data.  These divergences do not appear in
the energy spectrum.

Figure \ref{f:m20w019decomp} shows the time-dependence of the spectrum
in a chaotic irregular evolution, specifically $\mu=20,\sigma=0.19$
at $\epsilon=6.95$, which is in the chaotic region of the $t_H$ vs $\epsilon$
plot in figure \ref{f:m20w019}.  There is rapid energy transfer among modes,
including cascades out of and inverse cascades into mode numbers $j\leq 32$
over approximately a light-crossing time.  It is easy to imagine that
horizon formation might occur at any of the cascades of energy into higher
modes, leading to seemingly random jumps in $t_H$ as a function of amplitude.

\begin{figure}[!t]
\centering
\begin{subfigure}[t]{0.47\textwidth}
\includegraphics[width=\textwidth]{/Users/bradc/Documents/University_of_Manitoba/Thesis/PhD/Chapter1/figs/m0w11A101t71spect.pdf}
\caption{$\epsilon=1.01$}
\label{f:m0w11e101}
\end{subfigure}
\begin{subfigure}[t]{0.47\textwidth}
\includegraphics[width=\textwidth]{/Users/bradc/Documents/University_of_Manitoba/Thesis/PhD/Chapter1/figs/m0w11A102t71spect.pdf}
\caption{$\epsilon=1.02$}
\label{f:m0w11e102}
\end{subfigure}
\caption{Spectra at time $t\approx 71$ for $\mu=0,\sigma=1.1$ for the two
amplitudes given.  $\epsilon=1.01$ forms a horizon at $t_H\approx 71.1$,
$\epsilon=1.02$ at $t_H\approx 248.0$.}
\label{f:chaoticspect}
\end{figure}

Finally, the time-evolved energy provide another possible measure of
approximate thermalization in the dual CFT; namely, the spectrum should
approach an (exponentially cut-off) power law at thermalization.  In most
cases, this occurs shortly before horizon formation, but there are exceptions,
such as the late time behavior of initial data below the critical mass
for black hole formation in Einstein-Gauss-Bonnet gravity \cite{1608.05402}.
In the case of chaotic behavior, it is particularly interesting to know if
the spectra for similar amplitudes approach a power law at similar times
even if horizons form at very different times.  Figure \ref{f:chaoticspect}
shows the energy spectra for two amplitudes in the chaotic region of
the $t_H$ vs $\epsilon$ plot for $\mu=0,\sigma=1.1$.  Figure \ref{f:m0w11e101}
is the spectrum just before horizon formation for $\epsilon=1.01$,
while figure \ref{f:m0w11e102} is the spectrum at approximately the same
time for $\epsilon=1.02$, which is very long before horizon formation.  In
this example, we see that the spectrum does approach a power law for the
evolution that is forming a horizon, while the other evolution demonstrates
apparently exponential decay.  Therefore, this example suggests that
a power law spectrum may yield similar results to horizon formation as a
measure of thermalization in the dual CFT.


\section{Discussion}

We have presented the phase diagram of stability of AdS$_5$ against horizon
formation, treating the scalar field mass $\mu$ and width $\sigma$ of initial
data as free parameters.  In addition to mapping the location of the so-called
``island of stability,'' we have gathered evidence for two non-perturbative
phases on the ``shorelines'' of the island, the metastable and irregular phases.
While these must either exhibit stability (no collapse below some critical
amplitude) or instability (collapse at arbitrarily small but finite amplitude)
as the amplitude $\epsilon\to 0$, they are distinguished by their behavior
at computationally accessible amplitudes.  While perturbatively unstable
evolutions obey $t_H\propto\epsilon^{-2}$ as $\epsilon\to 0$, metastable
initial data follows $t_H\propto\epsilon^{-p}$ for $p>2$ over a range of
amplitudes.  The irregular phase is characterized by horizon formation times
$t_H$ that are not well described by a power law and sometimes exhibit
non-monotonicity or even chaos.  Both of these phases appear across the range
of $\mu$ values that we study and at both small- and large-width boundaries
of the stable phase.

At this time, it is impossible to say whether metastable initial data is
stable or unstable as $\epsilon\to 0$ (or if all metastable data behaves in the
same way in that limit).  Our numerical evolutions include cases in which
the lowest amplitudes jump either to metastable scaling with smaller $p$
or to evolutions that do not collapse over the timescales we study.
In many cases, too, the power law $t_H\propto\epsilon^{-p}$ with $p$ some
fixed value $>2$ is robust as we exclude larger amplitudes from our fit.
It is also possible that some of the metastable phase is stable in the
perturbative theory (ie, to $\epsilon^3$ order in a perturbative expansion)
but not at higher orders.

The irregular phase seems likely to be (mostly) stable at arbitrarily small
amplitudes based on our numerical evolutions, though we have not found a
critical amplitude for monotonic irregular initial data.  The irregular
phase includes the ``quasi-stable'' initial data described in
\cite{1304.4166,1508.02709}, which has a sudden increase then decrease in
$t_H$ as $\epsilon$ decreases as well as truly chaotic behavior.  In fact,
we have found evidence for weakly chaotic behavior at the jump in $t_H$
for non-monotonic initial data in the form of a small but nonzero Lyapunov
coefficient.  Both
non-monotonicity and chaos become stronger and more common at larger scalar
masses; however, we have also found chaotic behavior for the massless scalar.
To our knowledge, this is the first evidence of chaos in spherically
symmetric massless scalar collapse in AdS, which is particularly interesting
because there is only one physically meaningful ratio of scales, $\sigma$
as measured in AdS units.

Aside from the ultimate stability or instability of the metastable and
irregular phases, several questions remain.  For one, black holes formed in
massive scalar collapse in asymptotically flat spacetime exhibit a mass
gap for initial profiles wider than the Compton wavelength $1/\mu$
\cite{Brady1997}.  Whether this mass gap exists in AdS is not clear, and it
may disappear through repeated gravitational focusing as the field oscillates
many times across AdS; investigating this type of critical behavior will
likely require techniques similar to those of \cite{Santos-Olivan:2016djn}.
Returning to our phase diagram, the physical mechanism responsible for chaos
in the irregular phase is not yet clear.  Is it some generalization of the
same mechanism as found in the two-shell system?  Also, would an alternate
definition of approximate thermalization in the dual CFT, such as development
of a power-law spectrum, lead to a different picture of the phase diagram?
Finally, the big question is whether there is some test that could be performed
on initial data alone that would predict in advance its phase? So far, no
test is entirely successful, so new ideas are necessary.







\paragraph*{Acknowledgments} We would like to thank Brayden Yarish for help submitting jobs for the
$\mu=10$ evolutions.
The work of ND is supported in part by a Natural Sciences and Engineering
Research Council of Canada PGS-D grant to ND, NSF Grant PHY-1606654
at Cornell University, and by a grant from the Sherman
Fairchild Foundation.
The work of BC and AF is supported by the Natural Sciences
and Engineering Research Council of Canada Discovery Grant program.
This research was enabled in part by support provided by WestGrid
(www.westgrid.ca) and Compute Canada Calcul Canada (www.computecanada.ca).



\appendix
\section*{Appendices}
\addcontentsline{toc}{section}{Appendices}
\renewcommand{\thesubsection}{\Alph{subsection}}

\subsection{Convergence Testing}

Due to the large number of evolutions we have carried out, it is not
computationally feasible to test all of them for convergence.  Therefore,
we have checked several interesting cases from the irregular phase, which 
are the most curious.  These are carried out by evolving the initial data
with a base resolution $n=14$ and again at $n=15,16$ with commensurate
time steps, as described in \cite{1508.02709}.  In the cases indicated, we
evaluated the order of convergence at lower resolutions.  We remind the
reader that the order of convergence $Q$ is the base-2 logarithm of the ratio
of $L^2$ errors (root-mean-square over all corresponding grid points) between
successive pairs of resolutions.  We also note that the initial data is 
defined analytically, so $Q$ can appear poor at $t=0$ since the errors are
controlled by round off; in some cases, $Q$ is therefore undefined and 
not plotted.

\begin{figure}[!t]
\centering
\begin{subfigure}[t]{0.47\textwidth}
\includegraphics[width=\textwidth]{/Users/bradc/Documents/University_of_Manitoba/Thesis/PhD/Chapter1/figs/m05w1A112conv.pdf}
\caption{$n=14$ base resolution}
\label{f:m05w1e112conv}
\end{subfigure}
\begin{subfigure}[t]{0.47\textwidth}
\includegraphics[width=\textwidth]{/Users/bradc/Documents/University_of_Manitoba/Thesis/PhD/Chapter1/figs/m05w1A112conv11.pdf}
\caption{$n=11$ base resolution}
\label{f:m05w1e112conv11}
\end{subfigure}
\caption{Convergence results for $\mu=0.5$, $\sigma=1$, $\epsilon=1.12$
showing order of convergence $Q$ vs time for $\phi,M,A,\delta$ 
(blue, green, red, yellow respectively). Left: Resolutions $n=14,15,16$ used.
Right: Resolutions $n=11,12,13$ used.}
\label{f:m05w1convergence}
\end{figure}


First, we carried out convergence tests for mass $\mu=0.5$, width
$\sigma=1$, and amplitude $\epsilon=1.12$, which is monotonic irregular
initial data presented in figure \ref{f:m05w1}.  
This amplitude collapses with $t_H\sim 88$. Figure \ref{f:m05w1e112conv}
shows the ($L^2$ norm) order of convergence for the field variable 
$\phi$, the mass 
function $M$, and the metric functions $A,\delta$.  While the order of
convergence is initially poor and even negative, all these variables show
approximately fourth order convergence for times $t\gtrsim 70$.  The 
reason for the initially poor convergence is that the error between 
successive resolutions is already given by (machine limited) round off.
As a demonstration, we tested the order of convergence
with base resolution $n=11$, as shown in figure \ref{f:m05w1e112conv11}.
The variables show order of convergece $Q\gtrsim 3$ already at this resolution
for most of the evolution, losing convergence only for $t>80$, where we
see approximately 4th-order convergence in the $n=14$ resolution
computations.

Two of the authors have discussed the convergence properties of evolution for
the nonmonotonic irregular initial data with 
$\mu=20,\sigma=0.1,\epsilon=11.74$, which is in an amplitude region of 
increased $t_H$ surrounded by smaller values, in detail in \cite{1508.02709}.
In short, the variables $\phi,M,A,\delta$ all exhibit fourth order 
convergence, as does $\Pi^2(t,0)$, and the conserved mass actually has 6th 
order convergence.  

\begin{figure}[!t]
\centering
\begin{subfigure}[t]{0.31\textwidth}
\includegraphics[width=\textwidth]{/Users/bradc/Documents/University_of_Manitoba/Thesis/PhD/Chapter1/figs/m15w020.pdf}
\caption{Horizon formation}
\label{f:m15w020}
\end{subfigure}
\begin{subfigure}[t]{0.31\textwidth}
\includegraphics[width=\textwidth]{/Users/bradc/Documents/University_of_Manitoba/Thesis/PhD/Chapter1/figs/m15w020A742conv.pdf}
\caption{$\epsilon=7.42$}
\label{f:m15w020A742conv}
\end{subfigure}
\begin{subfigure}[t]{0.31\textwidth}
\includegraphics[width=\textwidth]{/Users/bradc/Documents/University_of_Manitoba/Thesis/PhD/Chapter1/figs/m15w020A740conv.pdf}
\caption{$\epsilon=7.40$}
\label{f:m15w020A740conv}
\end{subfigure}
\caption{Convergence results for $\mu=15$, $\sigma=0.2$.
Left: $t_H$ vs $\epsilon$.  Middle \& Right: order of convergence vs time 
for $\phi,M,A,\delta$ (blue, green, red, yellow respectively)
for indicated amplitudes.
}
\label{f:m15w020convergence}
\end{figure}

Initial data for $\mu=15,\sigma=0.2$ is also nonmonotonic, as shown in 
figure \ref{f:m15w020}.  While we have not analyzed all aspects of the 
convergence, we see from the remainder of figure \ref{f:m15w020convergence}
that $\phi,M,A,\delta$ exhibit convergent behavior at better than
second order for $\epsilon=7.42$
(figure \ref{f:m15w020A742conv}, second-largest value of $t_H$ in 
figure \ref{f:m15w020}) and $\epsilon=7.40$ (figure \ref{f:m15w020A740conv},
adjacent amplitude in figure \ref{f:m15w020}).  It is important to note that
the larger amplitude also has the larger horizon formation time, contrary
to the usual monotonic behavior.

\begin{figure}[!t]
\centering
\begin{subfigure}[t]{0.31\textwidth}
\includegraphics[width=\textwidth]{/Users/bradc/Documents/University_of_Manitoba/Thesis/PhD/Chapter1/figs/m0w11A102conv.pdf}
\caption{$\epsilon=1.02$}
\label{f:m0w11A102conv}
\end{subfigure}
\begin{subfigure}[t]{0.31\textwidth}
\includegraphics[width=\textwidth]{/Users/bradc/Documents/University_of_Manitoba/Thesis/PhD/Chapter1/figs/m0w11A101conv.pdf}
\caption{$\epsilon=1.01$}
\label{f:m0w11A101conv}
\end{subfigure}
\begin{subfigure}[t]{0.31\textwidth}
\includegraphics[width=\textwidth]{/Users/bradc/Documents/University_of_Manitoba/Thesis/PhD/Chapter1/figs/m0w11A100conv.pdf}
\caption{$\epsilon=1.00$}
\label{f:m0w11A100conv}
\end{subfigure}
\caption{Convergence results for $\mu=0$, $\sigma=1.1$ for listed amplitudes
showing order of convergence $Q$ vs time for $\phi,M,A,\delta$ 
(blue, green, red, yellow respectively); resolutions $n=12,13,14$.
}
\label{f:m0w11convergence}
\end{figure}

It is most crucial to validate the convergence of chaotic evolutions.
In table \ref{t:lyap}, we noted that the Ricci scalar at the origin has
nonzero Lyapunov exponent at almost the 2 sigma level for amplitudes
$\epsilon=1.02,1.01,1.00$ for $\mu=0,\sigma=1.1$.  We show the results of
convergence tests for these amplitudes in figure \ref{f:m0w11convergence};
because these are longer evolutions, we consider the convergence at the lower
resolutions $n=12,13,14$.  After a transient start-up period, these are all
convergent with $Q>2.5$ for all variables considered at all times; for most
of the time, the order of convergence is $Q>3.5$.  It is worth noting that
one of the amplitudes does not form a horizon through $t=500$.  





\begin{figure}[!t]
\centering
\begin{subfigure}[t]{0.31\textwidth}
\includegraphics[width=\textwidth]{/Users/bradc/Documents/University_of_Manitoba/Thesis/PhD/Chapter1/figs/m1w100.pdf}
\caption{Horizon formation}
\label{f:m1w100}
\end{subfigure}
\begin{subfigure}[t]{0.31\textwidth}
\includegraphics[width=\textwidth]{/Users/bradc/Documents/University_of_Manitoba/Thesis/PhD/Chapter1/figs/m1w100A115conv.pdf}
\caption{$\epsilon=1.15$}
\label{f:m1w100A115conv}
\end{subfigure}
\begin{subfigure}[t]{0.31\textwidth}
\includegraphics[width=\textwidth]{/Users/bradc/Documents/University_of_Manitoba/Thesis/PhD/Chapter1/figs/m1w100A114conv.pdf}
\caption{$\epsilon=1.14$}
\label{f:m1w100A114conv}
\end{subfigure}
\caption{Convergence results for $\mu=1$, $\sigma=1$. Left: $t_H$ vs $\epsilon$.
Middle \& Right: order of convergence $Q$ vs time for $\phi,M,A,\delta$ 
(blue, green, red, yellow respectively); resolutions $n=11,12,13$.
}
\label{f:m1w100convergence}
\end{figure}

Initial data with $\mu=1,\sigma=1$ is chaotic over a narrow range of 
amplitudes.  We have carried out convergence testing for amplitudes 
$\epsilon=1.15,1.14$, which are the two amplitudes with $t_H<100$ between
amplitudes with $t_H\gtrsim 150$ in figure \ref{f:m1w100}.  The order of
convergence was poor for these amplitudes in our initial tests with base
resolution $n=14$ because the error between resolutions was dominated by 
round-off, similar to the convergence tests we discussed above for
$\mu=0.5,\sigma=1$.  In subsequent tests with lower resolutions $n=11,12,13$,
we find an order of convergence $Q\sim 4$ for most of the evolutions
(and always $Q>3$). It is important to note again
that our evolutions exhibit convergence while showing horizon formation at 
a later time for a larger amplitude in this case.


\begin{figure}[!t]
\centering
\begin{subfigure}[t]{0.47\textwidth}
\includegraphics[width=\textwidth]{/Users/bradc/Documents/University_of_Manitoba/Thesis/PhD/Chapter1/figs/m20w019A695conv.pdf}
\caption{$\epsilon=6.95$}
\label{f:m20w019A695conv}
\end{subfigure}
\begin{subfigure}[t]{0.47\textwidth}
\includegraphics[width=\textwidth]{/Users/bradc/Documents/University_of_Manitoba/Thesis/PhD/Chapter1/figs/m20w019A692conv.pdf}
\caption{$\epsilon=6.92$}
\label{f:m20w019A692conv}
\end{subfigure}
\caption{Order of convergence vs time for $\phi,M,A,\delta$ (red, blue, green,
magenta) for $\mu=20,\sigma=0.19$ and indicated amplitudes. 
}
\label{f:m20w019convergence}
\end{figure}


Finally, we ran convergence tests for the chaotic initial data with
$\mu=20,\sigma=0.19$ for $\epsilon=6.95,6.92$, with $t_H\approx 65.5,40.8$
respectively.  As shown in figure \ref{f:m20w019convergence}, the simulations
are close to fourth order convergence for most of the evolution, but
there are periods where the order of convergence for evolution and constraint
variables becomes negative.  This of course leads to the concern that the
evolutions should have collapsed during those periods and extend into an
``afterlife'' evolution.  We have therefore
evolved these amplitudes through these regions (approximately $t=30-40$ for
$\epsilon=6.95$ and $t=18-30$ for $\epsilon=6.92$) at high resolution ($n=18$).
If the evolutions are truly in an afterlife, this higher resolution 
calculation may include horizon formation.  We do not observe this.  Another
tell-tale of would-be horizon formation is a decrease in the timestep size
by an order of magnitude or more followed by an increase.  We monitor
the timestep size every 500 timesteps through this evolution but do not
observe a decrease in timestep size by more than a factor of 2.  As a result,
we believe the values of $t_H$ found are reliable, though the reader may 
wish to consider them with some caution.  Nonetheless, we emphasize
that we have found convergent behavior for chaotic initial data at other
scalar masses.



%\bibliographystyle{JHEP}
%\bibliography{massive2}


\end{document}
