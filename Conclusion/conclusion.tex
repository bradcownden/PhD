\documentclass[../PhD.tex]{subfiles}

\begin{document}

%%%%%%%%%%%%%%%%%%%%%%%%%%%%%%%%%%%%%%%%%
%%%%%%%%%%%%%%%%%%%%%%%%%%%%%%%%%%%%%%%%%

\chapter{Conclusion}
\label{ch: conclusion}

In this thesis, we have addressed several facets of scalar field collapse in Anti-de Sitter spacetime as it relates to dynamical processes in strongly-coupled gauge theories. By considering a wide range of Gaussian initial data in AdS$_5$ in the nonperturbative (but small amplitude) regime, we have mapped out the islands of stability for a real scalar of mass $\mu$ and width $\sigma$. In addition to stable and unstable data, we have uncovered two new classes: those that exhibit metastability at finite amplitudes and scale like $t_H \propto \epsilon^{-p}$ for $p > 2$, and those that have non-monotonically increasing horizon formation times with decreasing amplitude. There is evidence for weakly chaotic evolution of data within the latter class, even for massless scalars. The root of the chaotic behaviour is yet to be determined, however there are indications that self-interaction occurs between regions of increased density produced by gravitational focusing. While similar chaotic evolutions have been observed between thin shells of infalling matter, this is the first time it has been observed in continuous initial data. To determine the ultimate fate of irregular and metastable data in the nonlinear theory as $\epsilon \to 0$, greater computing power is required; however, this is exactly the regime where the perturbative theory is applicable. Therefore, future progress may be made by a combination of perturbative and nonlinear evolutions.

Next, we examined perturbatively stable solutions through the Two-Time Formalism. Using renormalization flow methods, the secular growth of resonant contributions was absorbed into amplitude and phase variables so that the (massless) scalar field remained stable over perturbative timescales of $t \sim \epsilon^{-2}$. By introducing a quasi-periodic ansatz for the renormalized amplitudes, we found exact solutions to the truncated system. These solutions were tested for robustness against the choice of truncation value, where we found that much of the possible solution space was ruled out. This means that quasi-periodic solutions are potentially less able extend the stable, static TTF configurations than previously thought. If the islands of stability found in the nonlinear evolution are underpinned by solutions to the TTF description, a reduction in the space of solutions may mean that some generic initial data cannot be described by TTF solutions (quasi-periodic or otherwise). Future work may include examining the metastable and irregular data from nonlinear theory in the perturbative framework. Ideally, since evolution within the TTF theory is linear instead of nonlinear, scalar fields with amplitudes $\epsilon$ could be evolved with less computational cost over perturbative timescales, switching over to nonlinear evolution only when required. However, due to the symmetries of AdS, the perturbative theory is invariant under the rescaling $\phi(t) \to \epsilon^{-1} \phi (\epsilon^2 t)$, meaning that the time over which the perturbative theory is valid is not uniquely determined. An important contribution to the understanding of both the nonlinear and perturbative description of scalar field collapse would be to compare the evolutions in either scheme to establish some time scale for the applicability of the TTF. 

Finally, we extended the TTF description to include massive scalars, as well as time dependent boundary conditions. We saw that some of the symmetries that lead to the natural vanishing of two of the three resonance channels were broken by the presence of mass-dependent terms. In such cases, the renormalized amplitude and phase equations include contributions from multiple resonance channels. The inclusion of non-zero boundary terms requires the activation of non-normalizable modes. We explicitly calculated the $\mc O(\epsilon^3)$ source term for several choices of non-normalizable frequencies, which -- unlike normalizable modes -- are not constrained to be fully integer. Because energy enters the bulk through the boundary term, the evolution of such pumped systems will differ qualitatively from adiabatic evolution. The focus of future work will be constructing equilibrium solutions and analyzing their evolution. Like the more familiar massless scalars, there must exist inverse energy cascades to balance the transfer of energy to short length scales, thereby providing stability against collapse. This has yet to be addressed in the literature.

As research continues into the stability of Anti-de Sitter space, we will develop a better understanding not only of gravitational collapse, but also of dynamical processes in the strongly-coupled regimes of the dual theories. We have seen that even the simplest case, that of a minimally-coupled scalar field, has uncovered a surprising breadth of phenomena. Constructing a holographic dual to a realistic system, such as the high energy collision of heavy ions, introduces myriad complexities that may produce even more intriguing results. With each new discovery, we will gain a greater appreciation for curious relationship between gauge theories and higher-dimensional gravity.























%%%%%%%%%%%%%%%%%%%%%%%%%%%%%%%%%%%%%%%%%
%%%%%%%%%%%%%%%%%%%%%%%%%%%%%%%%%%%%%%%%%

\end{document}

%%%%%%%%%%%%%%%%%%%%%%%%%%%%%%%%%%%%%%%%%
%%%%%%%%%%%%%%%%%%%%%%%%%%%%%%%%%%%%%%%%%
