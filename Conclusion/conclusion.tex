\documentclass[../PhD.tex]{subfiles}

\begin{document}

%%%%%%%%%%%%%%%%%%%%%%%%%%%%%%%%%%%%%%%%%
%%%%%%%%%%%%%%%%%%%%%%%%%%%%%%%%%%%%%%%%%

\chapter{Conclusion}
\label{ch: conclusion}

In this thesis, we have addressed several facets of scalar field collapse in Anti-de Sitter spacetime as it relates to dynamical processes in strongly-coupled gauge theories. By considering a wide range of Gaussian initial data in AdS$_5$ in the nonperturbative (but small amplitude) regime, we have mapped out the islands of stability for a generic, real scalar. We have uncovered new classes of data: those that exhibit metastability at finite amplitudes and scale like $t_H \propto \epsilon^{-p}$ for $p > 2$, and those that have non-monotonically increasing horizon formation times with decreasing amplitude. There is evidence for chaotic evolution of data within the latter class, even for massless scalars, due to gravitational self-interaction. While similar chaotic evolutions have been observed between thin shells of infalling matter, this is the first time it has been observed with generic initial data.

Next, we examined perturbatively stable solutions through the Two-Time Formalism. Using renormalization flow methods, the secular growth of resonant contributions was absorbed into amplitude and phase variables so that the (massless) scalar field remained stable over perturbative timescales of $t \sim \epsilon^{-2}$. By introducing a quasi-periodic ansatz for the renormalized amplitudes, we found exact solutions to the truncated system. These solutions were tested against the choice of truncation value, and in doing so we found that much of the possible solution space was ruled out. Therefore, quasi-periodic solutions are less able extend the stable, static TTF configurations that previously thought.

Finally, 

Tie back into AdS/CFT \\

Talk about future work \\

%%%%%%%%%%%%%%%%%%%%%%%%%%%%%%%%%%%%%%%%%
%%%%%%%%%%%%%%%%%%%%%%%%%%%%%%%%%%%%%%%%%

\end{document}

%%%%%%%%%%%%%%%%%%%%%%%%%%%%%%%%%%%%%%%%%
%%%%%%%%%%%%%%%%%%%%%%%%%%%%%%%%%%%%%%%%%
