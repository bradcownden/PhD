\documentclass[../PhD.tex]{subfiles}

\begin{document}

%%%%%%%%%%%%%%%%%%%%%%%%%%%%%%%%%%%%%%%%%
%%%%%%%%%%%%%%%%%%%%%%%%%%%%%%%%%%%%%%%%%

\chapter{Perturbative Stability of Massless Scalars in AdS$_4$}

Having examined the collapse of massive scalars fields in AdS$_5$, we now wish to explore the perturbatively stable solutions for massless scalars. These solutions resist collapse on time scales of $t \sim \epsilon^{-2}$ and give analytic descriptions of the direct and inverse energy cascades that must be balanced for stability to be achieved. 

Using the Two-Time Formalism (TTF), renormalization flow equations are derived that absorb secular terms into renormalized integration constants in the first-order solution for the scalar field. These flow equations can be combined using a quasi-periodic (QP) ansatz to relate the amplitude and phases and lead to a system of $\jm + 1$ QP equations that relate the $\jm + 3$ unknowns. While the TTF theory technically involves an infinite sum of terms, by truncating the series to a finite $\jm$ value, numerical values for the amplitudes and phases can be calculated. How the truncation value affects the space of solutions, and whether these solutions continue to be valid during evolution, remains to be addressed.

%%%%%%%%%%%%%%%%%%%%%%%%%%%%%%%%%%%%%%%%%

\section{Contributions of Authors}

In this collaboration, QP solutions to \eqref{qp eqn} were found numerically through programs initially written by N.~Deppe, but later expanded and developed by myself. In particular, I developed code to achieve the tail fitting and seeding procedure detailed in appendix~\ref{app: seeding} that allowed for solutions to \eqref{qp eqn} to be developed for \jm values of several hundred -- almost an order of magnitude greater than the solutions previous found in the literature. Implementation of the high temperature perturbation method outlined in \ref{ssec: highT} was done using code I developed, as was the procedure of reoptimization that allowed for the high temperature solution to be projected back to the QP solution plane at various frequencies. Evolution of the solutions was based on numerical methods initially developed by N.~Deppe. All data management and analysis was done using programs written by myself.

Much of the numerical work for this project was done using the University of Winnipeg's tesla server, where CPU hours are not tracked. However, for larger systems increased computing power was required, which necessitated transferring all code to Compute Canada's new Cedar cluster. Once there, I used 5.43 CPU years' worth of computing power to run evolutions and analysis of the results. Finally, I wrote the paper, with input from the other authors, that appears here.

As is common for these types of projects, all members of the collaboration were equally involved in the interpretation of the data, as well as the late stages of editing. Authors are listed alphabetically and it is understood that all members contribute equally to the publication.

\newpage

%%%%%%%%%%%%%%%%%%%%%%%%%%%%%%%%%%%%%%%%%
%%%%%%%%%%%%%%%%%%%%%%%%%%%%%%%%%%%%%%%%%

\begin{center}
{\bf{\Large Paper Title}} \\
\bigskip
{\bf To Appear on \href{https://arxiv.org}{arxiv.org}} \\
\bigskip
\bigskip
Brad Cownden$^1$, Nils Deppe$^2$, and Andrew R.~Frey$^{3,4}$\\
\bigskip

$^1${\it Department of Physics \& Astronomy,\\ University of Manitoba\\
Winnipeg, Manitoba R3T 2N2, Canada \\ {\rm cowndenb@myumanitoba.ca}} \\
\vspace{0.1in}

$^2${\it Cornell Center for Astrophysics and Planetary Science and
Department of Physics,\\ Cornell University\\
122 Sciences Drive, Ithaca, New York 14853, USA \\ {\rm nd357@cornell.edu}} \\
\vspace{0.1in}

$^3${\it Department of Physics \& Astronomy,\\ University of Manitoba\\
Winnipeg, Manitoba R3T 2N2, Canada} \\
$^4${\it Department of Physics and Winnipeg Institute for Theoretical
Physics,\\ University of Winnipeg\\
515 Portage Avenue, Winnipeg, Manitoba R3B 2E9, Canada \\ {\rm a.frey@uwinnipeg.ca}}
\end{center}

\bigskip

We construct a family of perturbative solutions for massless scalar fields in AdS$_4$ using the \emph{Two-Time Formalism} (TTF) to high eigenmode numbers. We furthermore investigate the validity of \emph{quasi-periodic} (QP) solutions with high $j_{max}$ values and examine their stability to perturbations. Finally, check that TTF and QP solutions continue to satisfy the Einstein equation at times greater than $t \sim \epsilon^{-2}$ and compare these results to the full numerical solutions at low amplitude.

%%%%%%%%%%%%%%%%%%%%%%%%%%%%%%%%%%%%%%%%%

\section{Introduction}

The question of the nonperturbative stability of AdS$_{d+1}$ has been examined extensively, both as a question of mathematical physics and given its application to the AdS/CFT correspondence; see \cite{1708.05600} for a recent review. Beginning with the seminal work of \cite{1104.3702}, many works \cite{1108.4539, 1106.2339, 1110.5823, 1210.0890, 1510.02592} have demonstrated the generic instability of AdS$_{d+1}$ gravity minimally coupled to a scalar field in a variety of dimensions. The primary driver of the instability in the fully nonlinear system is the weakly turbulent flow of energy to short length scales; in the perturbative description, secular growth of resonant terms with high frequencies triggers the collapse \cite{1109.1825, 1306.0317, 1312.5544}. However, \cite{1303.3186, 1307.2875, 1403.5434} (and others) have shown that some initial conditions in asymptotically AdS spacetime resist gravitational collapse and therefore form islands of stability in the space of initial data.  The stable solutions within the island are variously known as oscillons or breathers for real scalars \cite{1104.3702,1210.0890,1303.3186,1503.07746}, boson stars for complex scalars \cite{1304.4166,1307.2875}, and geons for pure gravity \cite{1109.1825,1208.5772}.\footnote{Citations given for studies in asymptotically AdS space.} \cite{1508.02709, 1711.00454,1602.03535} have shown that the classification of initial data is more complex nonperturbatively, intriguingly finding evidence of chaos at the boundary between stable and unstable initial data. While past studies have mostly dealt with spherically symmetric collapse, an increasing amount of work is focused on removing this restriction \cite{1602.03890,1705.03065, 1706.06101}.

While the nonperturbative physics of AdS instability requires numerical study, a perturbative formulation should give insight into stability at low amplitudes.  In a naive perturbation theory, the fully resonant spectrum of eigenmodes of pure AdS leads to secular growth; this can be removed order by order by frequency shifts if the initial data consists of a single eigenmode but not for superpositions of eigenmodes \cite{1109.1825}.  If instead the amplitude and phase of each eigenmode are allowed to flow slowly, resummation of the perturbation theory leads to a ladder of coupled first-order ordinary differential equations describing the flow.  There are several equivalent methods to arrive at the flow equations: a ``two-time formalism'' (similar to a temporal gradient expansion for the amplitude and phase variables) \cite{1403.6471}, a renormalization-like formalism \cite{1407.6273,1412.3249}, time averaging \cite{1412.3249,1510.07836}, and keeping only resonant source terms \cite{1506.03519}.  (We will commonly refer to the perturbative theory as the TTF theory, for two-time formalism.)  A key feature of this perturbative theory is a scaling symmetry $\phi(t)\to \epsilon^{-1}\phi(\epsilon^2 t)$, so it is possible to divide out the amplitude of the scalar and describe the solution in terms of the ``slow time'' $\tau=\epsilon^2 t$.  Furthermore, the perturbative theory has conserved quantities beyond the total energy $E$, including a ``particle number'' $N$, which leads to inverse cascades in energy from higher eigenmodes to lower modes along with the expected direct cascades from low to high.  On the other hand, while the flow equations are significantly less computationally intensive than the full Einstein and Klein-Gordon equations, finding a solution requires truncating to a maximum eigenmode number \jm.

At a given mode truncation \jm, the TTF theory has stable quasi-periodic (QP) solutions with constant energy spectrum as described in \cite{1403.6471,1507.08261}, and other stable solutions orbit the QP solutions in phase space.  Since the amplitude scales out of the TTF, the QP solutions are described by ``temperature'' $T=E/N$; for fixed maximum mode number \jm, the maximum possible temperature is $d+2\jm$. The QP solutions are special in that the time-dependence of each mode is harmonic, so QP solutions satisfy algebraic equations; \cite{1507.08261} found low-temperature solutions to these equations directly.  To reach higher temperatures, \cite{1507.08261} perturbed low-temperature solutions by the addition of energy.  Our main concern in this work is the persistence of QP solutions, especially those at high temperatures, as $\jm$ increases since the full TTF theory takes $\jm\to\infty$.

****
LEFT THE REST ALONE, NEED TO DISCUSS ORGANIZATION AND METHODS WHEN WE'RE FINISHED

We show that high temperature QP solutions are very sensitive to truncation error and cannot be interpreted as physically relevant solutions. We then examine the time evolution of large $j_{max}$ QP solutions at all temperatures in both the perturbative theory and the full, nonlinear theory. {\bf [OTHER MAJOR GOALS HERE]}

This work is organized as follows: we begin in \S~\!\ref{sec: scalar in AdS} with a review of the linearized solutions for a minimally coupled, massless scalar field in AdS$_{d+1}$ and establish the renormalization flow equations that govern the time evolution of the amplitude and phase functions in the scalar field. In \S~\!\ref{sec: qp}, we find quasi-periodic solutions in AdS$_4$ by directly solving a set of algebraic equations, and discuss the viability of reaching new QP solutions through repeated application of a perturbative scheme. We then examine the time evolution of a wide range of QP solutions in \S\!~\ref{sec: time evolution} in both the linearized theory and the full, nonlinear system. We end with a discussion in \S~\!\ref{sec: discussion}. 


%%%%%%%%%%%%%%%%%%%%%%%%%%%%%%%%%%%%%%%%%

\section{Minimally Coupled Scalar Fields in AdS$_{d+1}$}
\label{sec: scalar in AdS}

Consider a spherically-symmetric, asymptotically AdS$_{d+1}$ spacetime with characteristic curvature $\ell$. Written in Schwarzschild-like coordinates, the metric in units of AdS scale is given by
\begin{align}
ds^2 = \frac{1}{\cos^2(x)} \left( -Ae^{-2\delta} dt^2 + A^{-1}dx^2 + \sin^2(x) d\Omega^{d-1}\right) \, ,
\end{align}
where the radius $x \in [0,\pi/2]$ and $-\infty < t < \infty$. A minimally-coupled, massless scalar field $\phi(t,x)$ is subject to the following Einstein and Klein-Gordon equations:
\begin{align}
\label{EEs}
G_{ab} + \Lambda g_{ab} &= 8\pi \left( \nabla_a \phi \nabla_b \phi - \frac{1}{2} g_{ab} (\nabla \phi)^2 \right) \\
\label{KG}
0 &= \frac{1}{\sqrt{-g}} \p_a \sqrt{-g} \, g^{ab} \p_b \phi \, .
\end{align}
The canonical equations of motion for the scalar field are
\begin{align}
\p_t \phi = A e^{-\delta} \Pi, \quad \p_t \Phi = \p_x ( A e^{-\delta} \Pi), \quad \text{and} \quad \p_t \Pi = \frac{\p_x \left(\Phi A e^{-\delta} \tan^{d-1} (x) \right)}{\tan^{d-1}(x)} \, ,
\end{align}
where the canonical momentum is $\Pi(t,x) = A^{-1}e^\delta \phi$ and $\Phi(t,x) \equiv \p_x \phi$ is an auxiliary variable. In terms of these fields, \eqref{EEs}-\eqref{KG} reduce to 
\begin{align}
	\label{EE const1}
	\p_x \delta &= - \left( \Pi^2 + \Phi^2 \right) \sin(x) \cos(x), \\
	\label{EE const2}
	\p_x A &= \frac{d - 2 + 2\sin^2(x)}{\sin (x)\cos(x)} (1 - A) - A \sin(x) \cos(x) (\Pi^2 + \Phi^2) \, .
\end{align}

%%%%%%%%%%%%%%%%%%%%%%%%%%%%%%%%%%%%%%%%%

\subsection{Linearized Solutions}

The linearized scalar field solutions come from expanding in terms of a small amplitude
\begin{align}
\label{eps expansion}
\phi(t,x) = \sum_{j=0}^\infty \epsilon^{2j+1} \phi_{2j+1}(t,x), \quad \! \! A(t,x) = 1 - \sum_{j=1}^\infty \epsilon^{2j} A_{2j}(t,x), \quad \! \! \delta(t,x) = \sum_{j=1}^\infty \epsilon^{2j} \delta_{2j} (t,x) . \hspace{-0.1in}
\end{align}
Under this expansion, the $\mc O(\epsilon)$ terms give the linearized equation of motion for the scalar field:
\begin{align}
\label{ttf eom}
\p^2_t \phi_1 + \hat L \phi_1 = 0 \quad \text{where} \quad \hat L_1 \equiv - \frac{1}{\tan^{d-1}(x)} \p_x \left( \tan^{d-1}(x) \p_x \right) \, .
\end{align}
The eigenvalues of $\hat L$ are simply $\omega^2_j = \left(d + 2j\right)^2$ and the eigenfunctions are
\begin{align}
\label{ttf eigens}
e_j (x) = k_j \cos^d (x) P^{(\frac{d}{2} - 1, \frac{d}{2})}_j \left( \cos(2x) \right) \quad \text{with} \quad k_j = \frac{2 \sqrt{j! (j + d - 1)!}}{\Gamma(j + \frac{d}{2})} \, .
\end{align}
Note the the normalizations are chosen such that $\hat L e_j = \omega^2_j e_j$ and 
\begin{align}
\langle e_i | e_j \rangle \equiv \int^{\frac{\pi}{2}}_0 dx \, \bar e_i e_j \tan^{d-1}(x) \, .
\end{align}
By expanding the scalar field functions in terms of the eigenbasis given in \eqref{ttf eigens} and substituting into \eqref{ttf eom}, we find that the time-dependent functions $c^{(2j + 1)}_n (t) = \langle \phi_{2j + 1}(t,x), e_n (x) \rangle$ satisfy $\ddot c_j^{(1)} + \omega^2_j c_j^{(1)} = 0$. The general solution for the scalar field is can then be written in terms of time-dependent amplitude and phase variables:
\begin{align}
\label{ttf phi}
\phi_1 (t,x) = \sum_{j=0}^\infty A_j (t) \cos \left(\omega_j t + B_j(t) \right) e_j (x) \, .
\end{align}

As discussed in \cite{1412.3249, 1407.6273, 1508.04943}, the integer nature of the mode frequencies mean that the spectrum is fully resonant. Unlike solutions such as oscillons, the resonant terms cannot be absorbed by a frequency shift and therefore result in \emph{secular} terms: resonant contributions that grow rapidly with time and induce collapse. These resonant terms appear at $\mc O (\epsilon^3)$ and can be expressed in terms of a source, $S(t)$:
\begin{align}
\ddot \phi_3 + \hat L \phi_3 = S \equiv 2 (A_2 - \delta_2) \ddot \phi_1 + ( \dot A_2 - \dot \delta_2 ) \dot \phi_1 + (A'_2 - \delta'_2) \phi'_1 \, ,
\end{align}
where $A_2$, $\delta_2$ are the leading-order contributions to the metric functions in \eqref{eps expansion} that are determined by the $\mc O(\epsilon^2)$ backreaction with the metric. Projecting onto the $e_j(x)$ basis, the source term ({\it i.e.}, resonant contributions) can be expressed in terms of the time-dependent coefficients
\begin{align}
\ddot c_j^{(3)} + \omega_j^2 c_j^{(3)} = S_j \, .
\end{align}  

To control the growth of secular terms, \cite{1412.3249} used resummation techniques to absorb singular contributions into the amplitude $A_j$ and phase $B_j$ of \eqref{ttf phi}. This also resulted in a set of conserved quantities: the energy of the system, $E$, and particle number, $N$. The simultaneous conservation of both $E$ and $N$ implied that weakly turbulent systems exhibit dual cascades of energy, providing a mechanism through which two-mode data could remain stable \cite{1412.4761}.

%%%%%%%%%%%%%%%%%%%%%%%%%%%%%%%%%%%%%%%%%

\subsection{Two-Time Formalism}

The Two-Time Formalism (TTF) describes the solution to \eqref{ttf eom} in terms of slowly-modulating amplitude and phase variables, $A_j$ and $B_j$, that are functions of the slow time $\tau = \epsilon^2 t$,
\begin{align}
\label{phi ttf}
\phi(t,x) = \epsilon \sum_{j=0}^\infty A_j (\epsilon^2 t) \cos \left(\omega_j t + B_j(\epsilon^2 t) \right) e_j(x) .
\end{align}
The next non-trivial order in the equations of motion include gravitational self-interactions of the scalar field, and provides source terms for $A_j$ and $B_j$. Following the time-averaging procedure of \cite{1407.6273} -- and using the resonance condition $\oi + \oj = \ok + \ol$ to eliminate one of the indices -- the $l^{th}$ amplitude and phase are given by
\begin{align}
\label{RN1}
-\frac{2\omega_l}{\epsilon^2} \frac{d A_l}{d t} &= \stackrel{l \leq i + j}{\sum_{i \neq l} \sum_{j \neq l}} S_{ij (i + j -l) l} A_i A_j A_{i + j - l} \sin \left( B_l + B_{i+j-l} - B_i - B_j \right) , \\
\label{RN2}
- \frac{2 \omega_l A_l}{\epsilon^2} \frac{d B_l}{dt} &= T_l A_l^3 + \sum_{i \neq l} R_{i l} A^2_i A_l  \nonumber \\
& \qquad + \stackrel{l \leq i + j}{\sum_{i \neq l} \sum_{j \neq l}} S_{ij (i + j -l) l} A_i A_j A_{i + j - l} \cos \left( B_l + B_{i+j-l} - B_i - B_j \right) \, .
\end{align}
The coefficients $T, R, S$ are calculated directly from integrals over the product of eigenmodes and contain som